\documentclass[a4paper,11pt]{article}

% Base packages
\usepackage{array}
\usepackage{xunicode,xltxtra,url,parskip}
\RequirePackage{color,graphicx}
% Page formatting
\usepackage{fontawesome}
\usepackage[big]{layaureo}

% Set up LaTeX logo
\usepackage{metalogo}
\setlogokern{La}{-0.05em}
\setlogokern{aT}{-0.05em}
\setlogokern{Te}{-0.1em}
\setlogokern{eX}{-0.04em}
\setLaTeXa{\raisebox{5em}{\scshape a}}

% Set up zmq logo
\newcommand{\zmq}{\O{}MQ}

% Setup link colors
\usepackage{hyperref}
\definecolor{linkcolour}{rgb}{0,0.2,0.6}
\hypersetup{colorlinks,breaklinks,urlcolor=linkcolour, linkcolor=linkcolour}

% Use multirow package for vertically joined rows
\usepackage{multirow}

% Enable configuring spacing in Skills
\usepackage{setspace}

% Enable intelligent spacing for using in dots
\usepackage{xspace}

% Load font
\usepackage{libertine}

% Configure title format
\usepackage{titlesec}
\titleformat{\section}{\large\scshape\raggedright}{}{0em}{}[{\titlerule[0.4pt]}]
\titlespacing{\section}{0pt}{0pt}{5pt}

% Variables
% Left column width
\newcommand{\lcolwidth}{2.2cm}
\newcommand{\lcolwidthinner}{2.1cm}
% Right column width
\newcommand{\rcolwidth}{16.2cm}


% Defines resume section environment
\newenvironment{rsection}[1]
  {
    \section{#1}
    \begin{tabular}{>{\raggedleft\arraybackslash}p{\lcolwidth}|p{\rcolwidth}}
   } {
    \\\multicolumn{2}{c}{} \\[-10pt]
    \end{tabular}
  }
% defines resume subsection header
\newcommand{\rheader}[2]{
    \multirow[t]{2}{*}{
        \begin{minipage}[t]{\lcolwidthinner}
            \begin{flushright}
                \textsc{#1}
            \end{flushright}
        \end{minipage}
    } & \textbf{#2}
}
% defines resume subsection subheader
\newcommand{\rdesc}[1]{
  \\[-2pt]&\small{\emph{#1}\vspace{1pt} }
}
% defines resume subsection line
\newcommand{\rline}[1]{\\& #1}
% defines resume subsection item
\newcommand{\ritem}[2][ •\hspace{3pt}]{\\[-2pt]& \footnotesize{#1#2}}
% defines resume subsubsection header
\newcommand{\rsubheader}[2]{\\[1pt]& \footnotesize{\textbf{#1} \textit{#2}}\\[-12pt]}
% defines resume subsubsection item
\newcommand{\rsubitem}[1]{\ritem[\hspace{6pt}•\hspace{4pt}]{#1}}
% Defines resume skills environment
\newenvironment{rskills}[1][Skills]
  {
    % \setstretch{0.75}
    \section{#1}
    \begin{tabular}{>{\raggedleft\arraybackslash}p{\lcolwidth}p{\rcolwidth}}
    } {
    \end{tabular}
  }
% defines resume skills section line
\newcommand{\rskill}[2]{\textsc{#1}:& \small #2 \\ & \\[-14pt]}
% defines resume subsection gap
\newcommand{\rskip}{\\\multicolumn{2}{c}{} \\[-5pt]}
% dot with spaces on the sides as appropriate
\newcommand{\rdot}{\xspace\hspace{0pt}•\hspace{3pt}\xspace}


% Begin document
\begin{document}

% Set up margins/origin
\hsize=7.5in \vsize=11in
\hoffset=-0.65in \voffset=-0.52in
% Output page size
\pdfpagewidth=8.5in
\pdfpageheight=11in
% Non-numbered pages
\pagestyle{empty}

% Meager attempt to slow down email spam scrapers
\newcommand{\at}{@}
\newcommand{\gmaildotcom}{gmail.com}

% Title
\begin{center}
     \Huge       Joseph Azevedo
  \\[2pt] \normalsize \href{mailto:jazevedo620\at\gmaildotcom}{jazevedo620\at\gmaildotcom}
    \rdot US Citizen \rdot (423) 284-1197 \rdot
\href{https://github.com/jazeved0}{\faGithub\ jazeved0} \rdot
    \href{https://jazevedo.me}{Portfolio: jazevedo.me} \\[6pt]
\end{center}
% Spacing
\vspace{11pt}


% Section: Education
\begin{rsection}{Education}
  \rheader{Jan 2022 -\\[-1pt] Current}{Georgia Institute of Technology,
    {\normalfont Atlanta, GA \hfill\  GPA: 4.0/4.0\ }}
  \rline{Master of Science, Computer Science \hfill Graduation date: Dec 2022}
  \vspace{2pt}
  \ritem[]{Concentration: Computer systems}
  \rskip
  \rheader{Jun 2018 -\\[-1pt] Dec 2021}{Georgia Institute of Technology,
    {\normalfont Atlanta, GA \hfill\  GPA: 4.0/4.0\ }}
  \rline{Bachelor of Science, Computer Science}
  \vspace{2pt}
  \ritem[]{Concentration: Networking \& Graphics}
\end{rsection}
% Spacing
\vspace{-4pt}


% Section: Skills
\begin{rskills}
  \rskill{Languages}  {Go, Rust, Python, C, C++, Bash, Java, C\#,
                      TypeScript, JavaScript, HTML/CSS, SQL}
  \rskill{Software}   {Git, Docker, Kubernetes, Nginx, gRPC, Linux, Windows, SQL (Postgres, MySQL)
                      NoSQL (MongoDB, Elasticsearch)}
  \rskill{Frameworks} {React, Flask, Express, Vue.js, jQuery}
  \rskill{Concepts}   {Containerization, Orchestration, Agile/SCRUM, Microservices,
                      Unit \& integration testing, CI/CD, Benchmarking}
  \rskill{Coursework} {Data structures, Algorithms, Databases, Networking,
                      Standard/Advanced operating systems, Cloud computing}
\end{rskills}
% Spacing
\vspace{7pt}


% Section: Work Experience
\begin{rsection}{Work Experience}
  % Job: Datadog software engineering intern
  \rheader{Aug 2021 -\\[-1pt] Dec 2021}{Software Engineering Intern}
  \rdesc{Datadog}
  \ritem{Designed a novel touchless monitoring daemon using \textbf{eBPF}
    to inspect \textbf{HTTPS} traffic created by \textbf{Go} services \textit{before encryption}}
  \ritem{Utilized \textbf{Linux eBPF}, Linux binary formats (\textbf{ELF}, \textbf{DWARF}),
    \& \textbf{Go} compiler/runtime internals to create proof-of-concept}
  \ritem{Developed code generation tooling to extract \textbf{Go} version-specific
    metadata from compiler artifacts, facilitating a wide range of supported \textbf{Go} versions}
  \rskip

  % Job: Stripe software engineering intern
  \rheader{May 2021 -\\[-1pt] Aug 2021}{Software Engineering Intern}
  \rdesc{Stripe}
  \ritem{Worked with another intern to develop a \textbf{Ruby} client library (incl.\ automated tests)
    for an internal config distribution platform}
  \ritem{Developed a last-mile caching sidecar in \textbf{Go} (incl.\ automated tests) to
    bypass limitations of \textbf{Ruby} multi-threading \& serve a local \textbf{HTTP API}}
  \ritem{Created separate \textbf{Kubernetes} and \textbf{Puppet} deployment
    configurations to facilitate integration with existing services}
  \rskip

  % Job: MathWorks software engineering intern
  \rheader{May 2020 -\\[-1pt] Aug 2020}{Software Engineering Intern}
  \rdesc{MathWorks}
  \ritem{Developed new features in a \textbf{Golang} microservice
    and a \textbf{React} dashboard, including unit and integration testing}
  \ritem{Designed a custom \textbf{Kubernetes} controller
    to work with internal framework and manage dynamic deployments}
  \ritem{Wrote design documentation and created proof of concept
    in \textbf{Go} investigating \textbf{Kubernetes} integration}
  \rskip

  % Job: CS 2340 TA
  \rheader{Aug 2019 -\\[-1pt] Dec 2020}{Senior Teaching Assistant}
  \rdesc{Georgia Institute of Technology \ {\normalfont |}\hspace{2pt}
    CS 2340 - Objects \& Design (Object-oriented design)}
  \ritem{Led a team of 6 other teaching assistants to prepare and deliver lectures
    over the course of the semester}
  \ritem{Graded project milestones and held office hours for students
    making a group project in \textbf{Java Swing} or \textbf{Python Flask}}
  \ritem{Created code style autograder scripts/workflow using \textbf{Python}
    for student projects used by 1,300+ students over 3 semesters}
\end{rsection}
% Spacing
\vspace{-3pt}


% Section: Leadership
\begin{rsection}{Leadership}
  % Position: GTE President
  \rheader{July 2019 -\\[-1pt] Aug 2020}{President}
  \rdesc{Georgia Tech Esports Club}
  \ritem{Led one of the largest student organizations at Georgia Tech
    with over \textbf{300 active members} and \textbf{30 competitive teams}}
  \ritem{Planned and led a team of \textbf{20 organizers} to host a regional
    collegiate LAN tournament
    (\href{https://web.archive.org/web/20201111230854/https://gamefest.gg/}{Gamefest})
    with over \textbf{400 participants}}
\end{rsection}
% Spacing
\vspace{-3pt}


% Section: Projects
\begin{rsection}{Projects}
  % Project: rAdvisor performance monitor
  \rheader{Feb 2020 -\\[-1pt] Dec 2021}{rAdvisor}
  \rdesc{Open-source system resource utilization tool for Docker \& Kubernetes
    {\normalfont \rdot
    \href{https://github.com/elba-docker/radvisor}{\faGithub\ elba-docker/radvisor}}}
  \ritem{Developed a high-performance, concurrent CLI tool in \textbf{Rust}
    that monitors \textbf{Linux} cgroups and polls the \textbf{Docker} daemon}
  \ritem{Conducted hundreds of distributed experimental workflows
    using \textbf{Python}/\textbf{Bash} to test overhead and consistency}
  \ritem{Wrote final report that details the software design, experimental procedure, and results
    \rdot \href{https://github.com/elba-docker/report}{\faGithub\ elba-docker/report}}
  \ritem{Integrated with performance monitoring toolkit as part of ongoing work
    as a \textbf{Research Assistant} at GT since Fall 2020}
\end{rsection}


\end{document}
